
% \setcounter{page}{0}

\chapter*{はじめに\hspace{-3mm}}

本書はプログラミングの方法ではなくプログラミング言語についての文書です。プログラムがどのように実行されるのかを理解しながら、プログラミング言語について学んでいきます。では、なぜ、プログラミング言語なのでしょう。

著者が働いている職場には優れた技術者はたくさんいます。ただし、それぞれ受けた教育など背景が異なるため、プログラムは書けても、その基礎となっている計算機科学(コンピュータサイエンス)の理解があやふやな人を、著者は多く見てきました。プログラミングに自信があるという人が、もう一歩先に進める道を示したいというのが、この文書を書き始めた動機です。

この文書を読むことで次の効果が得られることを期待しています。
\begin{itemize}
\item プログラミング言語とは何かを深く理解することで、プログラミングのレベルが上がる。
\item “この言語が良い!”と言う時に、シンタックス、ライブラリ、プログラミング言語固有の機能など、どのレベルの機能に言及しているのか区別できる。
\item プログラムがどのように実行されるか理解できる。
\item 関数型言語の中心となる概念を理解できる。
\item λ式とクロージャの違いが説明ができる。
\item 計算機科学の良書である通称SICP\footnote{Structure and Interpretaion of Computer Programs 2nd ed.: 訳本 計算機プログラムの構造と解釈 第2版}という本を読む準備ができる。
\item 友達に“関数型言語の処理系を作ったことがある”と言える ;-)
\end{itemize}
どれか一つにでも魅力を感じれば、この文書の読者の対象と言えるでしょう。

プログラミング言語を理解するために、これから$\mu$SchemeRという独自のプログ
ラミング言語を作成していきます。ものごとの本質を理解するには、その内部が
どうなっているのかを理解することが最も大切だと信じているからです。理解
するためには、実際に作ることが一番です。作成する言語は小さな関数型言語を
選びました。作成が簡単にも関わらず強力な機能を持っていること、通常の人
にはなじみが少ないパラダイムである関数型のプログラミング言語を理解する
ことにより、プログラミングの知識に幅を持てるようになるという理由からで
す。

内容が難しすぎそうと不安に思いますか。決してそんなことはありません。こ
の文書の知識の源になっている通称SICPという本は、MITの計算機科学での入門
レベルの講義に使われていました。著者も学部時代に研究室に配属されてまず読
まされた本です。そのくらい、計算機科学に携わる人には基本であり、だから
こそみなさんに知っていただきたい内容なのです。
たしかに、本書は読者として、ある程度のプログラミングをしたことのある人を想定
しています。ある程度がどの程度なのかの線引きは出来ませんが、なるべく、
興味を持ってくれた人に全員に理解してもらえるよう書いたつもりです。

$\mu$SchemeRを実現するためのプログラミング言語にはRubyを選びました。
Rubyは多くの人が親しんでいる手続き型言語であり、機能も強力なため、本質
的なことを簡潔に説明するのに役立ちます。Rubyに精通していなくとも、
Rubyの簡単な機能しか使いませんし、文章でも説明していきますので気負わず
読み進めてみて下さい。

本書では、なぜそれが必要なのかをできるだけ記載するようにしました。そ
れが本質を理解するための近道だと思うからです。正解だけを示すのは簡単で
すが、それがなぜ正解なのか、どうやって正解に行き着いたのか、問題は何だっ
たのかを読み取るのは容易ではありません。そこで、まずうまく動作しない例
を挙げ、問題を理解し、それを解決する手段を説明していきます。

本書で出てくるプログラムをコピー$\&$ペーストしていけば、実行できるよう
になっています。動作することを確認するくらいには役立つでしょう。ただし、
本当に理解したいのであれば、なるべく自分でプログラミングしてみてくださ
い。それもプログラムを見て理解した後は、そのプログラムを見ずに。
どこが理解できないでいるかが明確になるかと思います。

また、本書は末尾に示すとおり、クリエイティブ・コモンズ 表示 3.0 非移植
ライセンスの下に公開しています。改変、再配布をライセンスに従う範囲内で
認めていますので、後輩の教育に有効活用していただければこんなにうれしい
ことはありません。人に教えることが自分で学ぶことの一番の近道であること
を、著者はこの文書を書きながらつくづく実感しました。

実際、この文書は多くの人に助けられて作成されました。特に、堂阪真司さんには
Rubyプログラムの書き方を始め多くのことを教えていただきました。感謝しま
す。

前振りが長くなってしまいました。それでは、はじめましょう。
